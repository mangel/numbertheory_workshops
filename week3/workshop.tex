\documentclass{article}

\usepackage{amsmath, amsfonts, amssymb, amsthm}
\usepackage[utf8]{inputenc}
\usepackage[spanish]{babel}
\selectlanguage{spanish}
\usepackage[margin=0.5in]{geometry}

\title{Taller 2: Teoría de números}
\author{Miguel A. Gomez B.}

\begin{document}
	
	\maketitle
	
\paragraph{Ejercicio 1} Verifique, usando el principio del buen orden, que el conjunto $S = \{ 2x + 3y: x, n \in \mathbb{Z} \}$ tiene un elemento positivo mínimo y calcular este elemento.

\begin{proof}
Nos interesa demostrar la existencia de un número positivo por lo tanto construímos un nuevo conjunto que contiene los enteros positivos del conjunto $S$ (un subconjunto) y que llamaremos $S^+$, así:

\begin{equation}\label{eq:1}
S^+ = \{2x + 3y : x,y \in \mathbb{Z} \land 2x + 3y \ge 0\}
\end{equation}

Ahora debemos demostrar que $S$ es un conjunto no vacío para que se cumpla el principio del buen orden. Y en efecto es así, siempre y cuando $x\geq0$ e $y\geq0$, su suma también lo será, por la definición de suma de los números enteros positivos.

\paragraph{}
Hay dos casos más

\subparagraph{Caso I: $x<0$.} Si $2x < 0$, entonces para que se mantenga la construcción de $S^+$ se debe cumplir que $2x < 3y \geq |2x|$. Por ejemplo:
\begin{center}
	Si $x=-3$ e $y=2$, entonces
	\[
	2(-3) + 3(2) = -6 + 6 = 0
	\]
	Y siempre y cuando $3y = |2x| + n$ con $n \in \mathbb{N}$.
\end{center}

\subparagraph{Caso II: $y<0$.} Si $3y < 0$, entonces para que se mantenga la construcción de $S^+$ se debe cumplir que $3y < 2x \geq |3y|$. Por ejemplo (análogo al caso anterior):

\begin{center}
	Si $x=3$ e $y=-2$, entonces
	\[
	2(3) + 3(-2) = 6 - 6 = 0
	\]
	Y siempre y cuando $2x = |3y| + n$ con $n \in \mathbb{N}$.
\end{center}

Lo que demuestra que $S^+$ es no vacío. Luego por el principio del buen orden $S^+$ posee un elemento mínimo, lo cual implica que la parte positiva de $S$ tiene un mínimo y como vimos en los ejemplos anteriores, es el número $0$.
\end{proof}

\paragraph{Ejercicio 2} Prueba usando el principio de inducción, las fórmulas para $S_n$, $t_n$ y $T_n$.

\begin{equation} \label{eq:1}
	S_n = 1 + 3 +5 + \dots + 2n - 1 = n^2 
\end{equation}

\begin{equation}
	t_n = 1 + 2 + \dots + n - 1  + n = \frac{n(n+1)}{2}
\end{equation}

\begin{equation}
	T_n = t_1 + t_2 + \dots + t_n = \frac{n(n+1)(n+2)}{6}
\end{equation}

\begin{proof}(Por inducción para (\ref{eq:1})). La igualdad se cumple para $n=1$ pues $1 = 1^2 = 1$.
	
\subparagraph{Hipótesis de inducción.} Suponemos que la proposición es verdadera para $n=k$, es decir:

	\begin{equation}\label{eq:5}
		1 + 3 + 5 + \dots + 2k-1 = k^2
	\end{equation}

Ahora suponemos el siguiente $n = k + 1$ y por ende tenemos:

\[
	1 + 3 + 5 + \dots + 2k - 1 + 2(k+1) - 1 = (k + 1)^2\\
\]	

sin embargo por la hipótesis de inducción (\ref{eq:5}) tenemos ahora:

\[
	k^2 + 2(k+1) - 1 = (k + 1)^2\\
\]

Manipulando algebráicamente la expresión $k^2 + 2(k+1) - 1$ deberíamos obtener $(k + 1)^2$:

\begin{align*}
	k^2 + 2(k+1) - 1 &= k^2 + 2k + 2 - 1 \\
	&= k^2 + 2k + 1\\
	&= (k + 1)^2
\end{align*}

Por lo tanto hemos demostrado que si la proposición es correcta para $n = k$, es correcta para $n=k+1$. Entonces, por el principio de inducción, la fórmula es válida para todo $n \in \mathbb{N}$.

\end{proof}

	
\end{document}

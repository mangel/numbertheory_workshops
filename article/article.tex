\documentclass{article}

\usepackage{amsmath}
\usepackage{amsfonts}
\usepackage{amssymb}
\usepackage{amsthm}
\usepackage{amsrefs}
\usepackage[utf8]{inputenc}
\usepackage[spanish, mexico]{babel}
\usepackage{dirtytalk}

\title{Sobre la construcción de los números}
\author{Miguel Angel Gomez Barrera}

\begin{document}
\maketitle

\begin{center}\textbf{Abstract}\end{center}
\paragraph{} In English

\begin{center}\textbf{Introducción}\end{center}

\paragraph{} ¿Qué es un número? Las primeras nociones nos llevan a que es una representación de una cantidad, a fin de cuentas los utilizamos para medir y para contar, desde el número de miembros de nuestra familia, hasta la distancia del sol a la tierra, son una construcción valiosa para la humanidad, tal es la fascinación de algunas personas por estos objetos que tenemos representaciones del concepto de nada y del todo, ambos tienen un símbolos 0 y $\infty$, el primero atribuído a Matemáticos indios y el segundo debido a los griegos por primera vez. Sin embargo, tenemos numeros naturales, enteros, racionales y reales ¿Por qué? podríamos decir que es en base a dos necesidades, una operativa, y que más adelante veremos que algunas operaciones requieren de otro tipo de números, una consecuencia del proyecto de Hilbert (en el siglo XIX)\footnote{Hay una razón de fondo que intentaré explicar en base al capítulo 53 de \cite{morris_kline_pensamiento_1972}, esta labor se debe a la aparición de postulados que eran contradictorios (el postulado de las paralelas por ejemplo), durante la época se tenía a las matemáticas como una verdad inquebrantable, al evidenciar contradicciones, muchos matemáticos se dividieron e intentaron construir unas bases axiomáticas que impidieran la aparición de contradicciones. Sin embargo el sueño de Hilbert nunca llegó a completarse debido a los teoremas de incompletitud de Gödel, que demuestran que no es posible construir matemáticas sin que aparezcan contradicciones.
}.

\section{Los números naturales}
Iniciamos con la noción de los números naturales, aparecen como una necesidad básica (y "natural"): contar. Estos números tienen propiedades muy interesantes, como los números primos. Una de sus formas de axiomatización\footnote{Axiomatizar es formalizar una teoría a partir de verdades básicas, de esta manera se puede "garantizar" que no van a aparecer errores o contradicciones en toda la teoría.} mayormente aceptada es la de Peano.

\subsection{Los Axiomas de Peano}

Los axiomas de Peano\footnote{Giusseppe Peano (1858-1932). Fue un matemático italiano muy talentoso y de orígenes muy humildes, nacido en una familia de campesinos, su talento fue descubierto gracias al hermano de su madre, un clérigo y abogado. Que patrocinó sus primeros estudios.} son los siguientes\cite{tnumprincip_2004}

\paragraph{Axioma 1}. Hay un elemento especial 0 $\in \mathbb{N}$. \footnote{¿Desde dónde empezamos a contar? Esta pregunta dividió a los matemáticos, naturalmente empezamos desde el 1, pero en algunos casos el cero es un elemento muy conveniente a tener, lo mas "sano" entre matemáticos al axiomatizar, es declarar con antelación si se incluye o no el 0 en una demostración}.
\paragraph{Axioma 2}. Para todo $n \in \mathbb{N}$ existe un único elemento $n^+ \in \mathbb{N}$ llamado el sucesor de $n$.
\paragraph{Axioma 3}. Para todo $n \in \mathbb{N}$, $n^+ \neq 0$.
\paragraph{Axioma 4}. Si $n$, $m \in \mathbb{N}$ y $n^+ = m^+$ entonces $m = n$.
\paragraph{Axioma 5}. Si $S$ es un subconjunto de $\mathbb{N}$ tal que:
\begin{enumerate}
    \item $0 \in \mathbb{N}$
    \item $n^+ \in S$ siempre que $n \in S$, entonces $S = \mathbb{N}$
\end{enumerate}

\paragraph{} Con estos axiomas es posible construir todo el conjunto de naturales, el primero por ejemplo, ya inclye al número $0$ en los naturales y según la notación hay un número $0^+$. Vamos a asumir que $n^+ = n + 1$,  de esta manera podemos decir que $0^+ = 0 + 1 = 1$, de esta misma manera $1^+ = 1 + 1 = 2$ (que es la nocion del axioma 2). Todos los demás números se pueden expresar de esta manera, pero podrá haber algún número dentro de los naturales que cumpla ¿$n^+ = 0$? Supongamos que existe, ello implicaría que $0 = n + 1$ lo cual contradice el axioma 3, y justamente por esto existe, de manera implícita, indica que no existe ningún número que sea el sucesor del 0 (otra manera de verlo es que el 0 es un número especial que no puede recrearse mediante la operación de sume, por lo que de cierta manera se asume su existencia).

\paragraph{} Hasta ahora con todo lo anterior podemos construir todos los números naturales, sin embargo el axioma 4 y 5, tienen mayor relación con las demostraciones de propiedades de los naturales, con las operaciones, pero el axioma 4 en particular permite demostrar propiedades para todos los números si se cumplen sus dos premisas, algo que es muy útil y práctico en la labor de demostrar, por ejemplo las propiedades de las operaciones.

\paragraph{} Gracias a la curiosidad y cierta genialidad, es que a lo largo de la historia se han encontrado propiedades que son interesantes "matemáticamente" siendo la de mayor interés la de los números primos. Pero antes de entrar aquí debemos revisar las operaciones básicas del conjunto, la suma y la multiplicación.

\subsection{La suma}

Conocida también como adición. Su noción está relacionada con contar colecciones de cosas y se representa con el símbolo "+" que se lee mas.

La suma tiene varias propiedades

\begin{itemize}
    \item Es asociativa, es decir (a + b) + c = a + (b + c), no importa el la forma en la que se asocian los valores de una suma, el resultado va a ser el mismo.
    \item Es Conmutativa, es decir m + n = n +m, No importa el orden en el que se realiza la suma.
    \item Tiene un elemento neutro.  0 + a = a + 0 = 0. El cero es el elemento neutro.
    \item Es cerrada, es decir a + b = c, donde c siempre va a ser un número natural.
\end{itemize}

En matemáticas las proposiciones deben ser demostradas, vamos a hacer algunas, pero primero necesitamos la definición de suma.

\paragraph{Definición 1}\cite{tnumprincip_2004}\label{def:sum} Las siguientes ecuaciones definen la adición en $\mathbb{N}$. Para todo $n,m \in \mathbb{N}$:

$$m + 0 = m,$$
$$m + n^+ = (m + n)^+$$

Utilizando esta definición y los axiomas de Peano demostraremos algunas de sus propiedades.

\paragraph{Teorema.} La suma es asociativa, es decir para todo $a,b,c \in \mathbb{N}$ $(a + b) + c = a + (b + c)$.

\paragraph{Demostración} \textit{Por inducción} (utilizando el axioma 5)

Definimos un conjunto $S = \{c \in \mathbb{N} | (a + b) + c = a + (b + c) para todo a,b \in \mathbb{N}\}$

\paragraph{} $0 \in \mathbb{N}$ porque por la definición de suma
$$ (a + b) + 0 = a + (b + 0) = a + b$$

Ahora construimos la hipótesis de inducción y que asumimos verdadera
\paragraph{Hipótesis de inducción} Suponemos que $c \in \mathbb{N}$ luego es verdadero que 
$$(a + b) + c = a + (b + c)$$

Luego para el sucesor deberíamos llegar a que se cumple que $(a + b) + c^+ = a + (b + c^+)$, utilizamos el lado izquierdo de la igualdad:

    $$(a + b) + c^+ = ((a + b) + c)^+$$

esto es por la definición de suma, además esta forma es conveniente y equivalente a la hipótesis de inducción por ello ahora decimos que

$$((a + b) + c)^+ = (a + (b + c))^+$$

Aún no hemos llegado a la forma $a + (b + c^+)$ pero encontramos que por la definición de suma:

$$(a + (b + c))^+ = a + (b + c)^+$$

y por la misma definición de suma

$$a + (b + c)^+ = a + (b + c^+)$$

Asumiendo la hipótesis de inducción como verdadera y dado que al evaluar la propiedad en el sucesor, hemos demostrado que la propiedad se cumple para cualesquiera números naturales. $\blacksquare$\footnote{El símbolo $\blacksquare$ significa que la prueba ha concluído y hemos evaluado el valor de verdad de la proposición inicial.}

\paragraph{} Revisemos un poco lo que hemos hecho en las demostraciones anteriores, definimos unos primeros principios y que asumimos como ciertos, por ejemplo $a + 0 = a$, estos no requieren una demostración, en cierto modo son verdades que se asumen sin demostración. Luego aprovechamos la propiedad constructiva que discutimos acerca de los números naturales (que se construyen uno en base a otro desde el 0), únicamente que en este ejemplo no utilizamos ningún ejemplo numérico sino literales, que representan cualquier elemento de los números naturales, aquí fue donde aplicamos el axioma 5 y encontramos que las propiedades se cumplen, es una forma un poco más rigurosa pero permite dar un cierto grado de validez a lo que se está construyendo. Realicemos algunas demostraciones más.

\paragraph{} Para demostrar la propiedad de conmutatividad primero debemos demostrar lo siguiente

\paragraph{Lema} Para todo $a \in \mathbb{N}, 0 + a = a$ \footnote{Parece evidente, pero recordemos que la definición inicial que conocíamos únicamente dice que $a + 0 = a$ y dado que aún no hemos demostrado que éste lema sea verdadero y que vamos a utilizarlo de aquí en adelante, el rigor de la construcción en matemáticas exige demostrarlo.}

\paragraph{Demostración} Definimos un conjunto $S = \{a \in \mathbb{N}| 0 + a = a \}$

\paragraph{} $0 \in \mathbb{N}$, porque $0 + 0 = 0$ por la definición de suma.

\paragraph{Hipótesis de inducción} suponemos que $k \in \S$ y se cumple que $0 + k = k$. por lo que al aplicar la propiedad al sucesor esperaríamos obtener $k^+$.

\paragraph{}Ahora al evaluar en el sucesor la proposición, encontramos por la definición de suma que:

$$ 0 + k^+ = (0 + k)^+$$

y al aplicar la hipótesis de inducción tenemos que en efecto era lo que esperábamos

$$(0 + k)^+ = k^+$$

Por lo tanto la proposición se cumple para todos los naturales. $\blacksquare$

\paragraph{} Requerimos ahora de un lema adicional.

\paragraph{Lema} Para todo $a,b \in \mathbb{N}, a^+ + b = (a + b)^+$.

\paragraph{Demostración}. Construímos el conjunto $S =  \{ b \in \mathbb{N}| a^+ + b = (a + b)^+ \}$

\paragraph{} $0 \in \mathbb{N}$, porque $a^+ + 0 = a^+$ por la definición de suma.

\paragraph{Hipótesis de inducción} Suponemos que $k \in S$ y se cumple que $a^+ + k = (a + k)^+$. Por lo que al aplicar la propiedad al sucesor esperaríamos obtener $(a + b^+)^+$.

\paragraph{}Ahora al evaluar en el sucesor la proposición encontramos por la definición de suma que:

$$ a^+ + k^+ = (a^+ + k)^+$$

y al aplicar la hipótesis de inducción tenemos,

$$(a^+ + k)^+ = (a + k)^{++}$$

Y por la definición de suma finalmente obtenemos

$$(a + k^+)^+$$

Por lo tanto la proposición se cumple para todos los naturales. $\blacksquare$

\paragraph{} Con lo visto hasta aquí puede creerse que hay una fórmula para realizar demostraciones, nada más lejos de la realidad, con los resultados descubiertos por Gödel, existen proposiciones que sabemos no es posible de demostrar y aún mas intrigante no podemos saber cuales son. Muchos problemas como la distribución de los primos se creen que están en este conjunto de problemas que no podrán demostrarse\footnote{Es decir pueden matemáticos en estos momentos trabajando en problemas a los que nunca van a encontrar una solución.}. Por ahora veamos un ejemplo de una demostración un poco más compleja.

\paragraph{La suma es conmutativa} Sean $a,b \in \mathbb{N}, m + n = n + m$.

\paragraph{Demostración} Definimos un conjunto $S = \{b \in \mathbb{N}|a + b = b + a\text{, para todo } a \in \mathbb{N}\}$.

\paragraph{} $0 \in S$, por el lema que demostramos anteriormente donde $0 + a = a$ y por la definición de suma sabemos que $a + 0 = a$, luego $0 + a = a + 0$.

\paragraph{Hipótesis de inducción} Suponemos que $k \in S$ y se cumple que $a + k = k + a$. Por lo que al aplicar la propiedad al sucesor esperaríamos obtener $k^+ + a$.

\paragraph{} Ahora al evaluar en el sucesor la proposición encontramos por la definición de suma que:
$$a + k^+ = (a + k)^+$$

Por la hipótesis de inducción obtenemos

$$(a + k)^+ = (k + a)^+$$

y por el lema demostrado anteriormente donde $a^+ + b = (a + b)^+$ obtenemos

$$(k + a)^+ = k + a^+$$

y al aplicar el axioma 4 tenemos que

$$k + a = a + k$$

Que era lo que queríamos demostrar, por lo tanto la propiedad se cumple para los naturales. $\blacksquare$

Como vimos construímos esta verdad con lo que conocemos hasta ahora. Veamos la multiplicación.

\subsection{La multiplicación}
Conocida también como producto, su noción también es muy familiar, consiste en repetir un numero de sumas, se representa como una equis 'x' o simplemente se 'agrupan' los números $ab$.

La multiplicación tiene varias propiedades

\begin{itemize}
    \item Es asociativa, es decir (ab)c = a (bc), no importa el la forma en la que se asocian los valores de una multiplicación, el resultado va a ser el mismo.
    \item Es conmutativa, es decir mn = nm, no importa el orden en el que se realiza la multiplicación.
    \item Tiene un elemento neutro.  1a = a1 = a. El uno es el elemento neutro.
    \item Es cerrada, es decir ab = c, donde c siempre va a ser un número natural.
    \item Es distributiva respecto a la suma, es decir a(b + c) = ab + ac.
    \item Elemento cero. 0a = 0. multiplicar cualquier número por 0 siempre será 0.
\end{itemize}

\paragraph{}Las siguientes ecuaciones definen la multiplicación en $\mathbb{N}$. Para todo $m,n \in \mathbb{N}$\cite{tnumprincip_2004},

$$m0=0,$$
$$mn^+ = mn + m$$

Como todo número natural distinto de cero es el sucesor de otro número sucesor natural, la operación está bien definida.

Demostremos esta propiedad.

\paragraph{Teorema} La multiplicación es distributiva. Sean $a,b,c \in \mathbb{N}$, a(b + c) = ab + ac.

\paragraph{Demostración} \textit{Por inducción}. Haremos uso de otros teoremas y definiciones mencionadas anteriormente.

\paragraph{} Construrímos el conjunto $S = \{ c \in \mathbb{N}| a(b+c) = ab + ac \text{ para todo } a,b \in \mathbb{N}\}$

\paragraph{} Verificamos que $0 \in \mathbb{N}$, y por ende queremos demostrar que $a(b + 0) = ab + a0$. Por la definición de suma tenemos que

	$$a(b + 0) = ab$$

ahora por la propiedad del elemento neutro podemos decir que

$$a(b + 0) = ab = ab + 0 $$

y por la definición de multiplicación obtenemos

$$ab + 0 = ab + a0$$

Que era lo que queríamos demostrar.

\paragraph{Hipótesis de inducción} Asumimos que la propiedad se cumple para cualquier $k \in \mathbb{N}$:

$$a(b + k) = ab + ak$$

y verificamos que se cumpla la propiedad para el sucesor, luego deberíamos verificar que:

$$a(b + k^+) = ab + ak^+$$

Por la definición de múltiplicación tenemos que

$$ab + ak^+ = ab + ak + a$$

Por la propiedad asociativa podemos agrupar convenientemente esta ecuación

$$ab + ak + a = (ab + ak) + a$$

Utilizamos ahora la hipótesis de inducción, tenemos

$$(ab + ak) + k = a(b + k) + a$$

Y por la definición de multiplicación tendríamos que 

$$a(b + k) + a = a(b + k)^+$$

Por la definición de suma tenemos efectivamente que 

$$a(b + k)^+ = a(b + k^+)$$

Y que era lo que queríamos demostrar.$\blacksquare$

Hasta este momento lo que hemos visto hasta aquí podemos hacer varias construcciones sin salirnos del conjunto\footnote{Los conjuntos aún no han sido definidos completamente, aún no hay una definición universal entre los matemáticos de conjunto debido a las paradojas que se derivan. Fue en el siglo XIX que gracias a Cantor y algunos de sus seguidores se estableció la teoría de conjuntos como una de las axiomatizaciones ampliamente aceptadas entre los matemáticos. Aunque hoy en día el programa de Langland es el que se está encargando de construir una teoría unificada de las matemáticas, su nombre es en honor al matemático que a fecha de hoy se encuentra trabajando en esta teoría Robert Langlands.} de los naturales pero hay operaciones que arrojan resultados fuera de su conjunto, como la resta y la división\footnote{Aún no definimos estas dos operaciones, vamos a estudiar primero su necesidad y su relación con la construcción de otros conjuntos}, por lo que podríamos decir que en cierta manera la construcción de estos conjuntos nace como una necesidad de formalizar los resultados de algunas operaciones. Veamos algunos ejemplos.

\paragraph{Ejemplo} Sean a y b dos números naturales y $a > b$ tenemos que al sumarlos sin importar el orden, siempre vamos a obtener un número natural, pero al restarlos tenemos dos casos.

\subparagraph{Caso I} (a - b). Como a es mayor que b, al "quitar" b unidades de a, obtendremos un número mayor a 0. Y por ende se encuentra en el conjunto de los naturales.

\subparagraph{Caso II} (b - a). Si lo definimos en el conjunto de los naturales, encontraremos que el número (b - a) no existe en los naturales, porque su valor será menor a 0.

\section{Los números enteros}

Ahora podemos comprender mejor la necesidad de un conjunto mas amplio que los naturales. En escencia uno que contenga los valores menores a 0. se le denota con el símbolo $\mathbb{Z}$.

\subsection{La resta}

Conocida también como diferencia, se representa con el símbolo menos '-'. $a - b$, se lee a menos b. Y en su esencia es una suma entre enteros. Luego tenemos varios casos

\paragraph{Caso I} (a + b). Si uno de los dos números es menor a 0. En cuyo caso podemos obtener un número natural o un número menor a 0. Ello dependerá de la magnitud de un número respecto a otro.

\subparagraph{Ejemplo} $a + b$, digamos que $a < 0$, luego si $|a| > b$ tenemos que $|a| = b + k$ donde $k>0$ y ahora si evaluamos en los naturales $b + k > b$ luego $k$ sería 'Lo que le falta a $b$ para ser $a$' (la diferencia). Ahora numéricamente, 

$$-7 + 5$$
$$|7| > 5$$

Luego $|7| = 5 + k$, para que esto se cumpla necesariamente $k = 2$, $2$ es lo que le falta a $5$ para ser $7$, ahora en magnitud su diferencia es 2 y en éste cuando $a>b$ decimos que su resultado es 

$$-7 + 5 = -2$$

es decir

$$a + b = -c$$

\paragraph{Caso II} - a - b. Para este caso tenemos que $-a - b = -(a+b)$. Y siempre obtenemos valores negativos.

\subsection{Multiplicación}

Respecto a la resta, se utilizan las mismas definiciones y propiedades de los naturales, únicamente tendremos que añadir a nuestro repertorio de definiciones

$$a(-b) = -ab$$
$$(-a)(-b) = ab$$

\paragraph{} Lo que hemos visto hasta ahora nos da una noción de operación inversa, al sumar, añadíamos cosas a otras cosas, al restar, quitamos cosas a otras cosas, pero y ¿al multiplicar? De ello surgió otro enigma

\paragraph{}Al dividir lo que hacemos es contar cuantos conjuntos de cosas hay en un conjunto de cosas, luego si queremos dividir 4 entre 2, decimos que hay 2 dos en 4, es decir

$$4 = 2(2)$$

pero que hay de hallar lo anterior entre 5 y 7 y viceversa, y aún mas que hay de hacer lo anterior con el número 0.

\paragraph{Divisibilidad y dividir} Nace la noción de divisibilidad, de cierto modo hemos visto que hay resultados de divisiones que existen en los enteros pero hay otros que no, por ejemplo

$$\frac{9}{5}$$

No hay un único número\footnote{No hay un único dentro de $\mathbb{Z}$, pero hay una forma de representar estos números dentro de los enteros y viene dado por el algoritmo de la división. En esencia al utilizar esta construcción podemos decir que 7 = 5(1) + 2, literalmente $a = bq + r$, donde q(cociente) y r(residuo), deben satisfacer que $0 \leq r<b$.} dentro de $\mathbb{Z}$ que represente el resultado de esta operación. Por ello al igual que ocurrió en la resta, se define un nuevo conjunto el conjunto de los números racionales $\mathbb{Q}$.

\section{Los números racionales}

Un número racional es de la forma:

$$\frac{a}{b}$$

Donde $a, b \in \mathbb{Z}$ y $b \neq 0$ necesariamente\footnote{Más adelante veremos la razón de esto. Por ahora asumiremos que estos números no existen.}.

En este conjunto podemos ubicar, a los números resultado de las operaciones fuera del conjunto de los enteros y que vimos anteriormente\footnote{Su aparición tiene un gran recorrido histórico. Los babilonios por ejemplo sólamente utilizaban fracciones con base 60; los romanos en base 12. Los egipcios únicamente utilizaban fracciones con cualquier número como denominador pero no admitían ninguno que fuese diferente a 1 como numerador y en lugar de escribir $\frac{2}{5}$ escribían la expresión $\frac{1}{3} + \frac{1}{15}$. La notación actual parece provenir desde el año 1202 y que aparece en el 'Liber Abaci' de Leonardo de Pisa\cite{patino_duque_1977}}, respecto a la suma debemos añadir que hay dos casos:

\paragraph{Caso I:} Cuando las fracciones tienen un número igual. En este caso, se define:

$$\frac{a}{b} + \frac{c}{b} = \frac{a + c}{b}$$

E igualmente se debe satisfacer que $b \neq 0$. $a, b, c \in \mathbb{Z}$.

\paragraph{Caso II:} Cuando las fracciones tienen un denominador diferente.

$$\frac{a}{b} + \frac{c}{d} = \frac{ad}{bd} + \frac{cb}{db} = \frac{ad + cb}{db}$$

\paragraph{} Y ahora, respecto a la multiplicación:

$$\frac{a}{b} \times \frac{c}{d} = \frac{ac}{bd}$$

\paragraph{} Y ahora, respecto a la división:

$$\frac{a}{b} \div \frac{c}{d} = \frac{a}{b} \times \frac{d}{c} = \frac{ad}{bc}$$

\subsection{Dividir por 0}

Anteriormente afirmamos que los números de la forma $\frac{a}{0}$ no existen, ahora veremos por que.

\paragraph{}Sabemos que la multiplicación define que $a0 = 0$, cuando $a$ es cualquier número, y por la división definimos (en $\mathbb{Q}$) que existe un único número tal $a \times b = c$. Realizaremos una adaptación de la demostración en\cite{brink_1933}.

\paragraph{Demostración} Existen dos casos.

\paragraph{Caso I:} Suponga que $a \neq 0$ e intente encontrar un número $b$ tal que

$$0b = a$$

Pero por la definición que conocemos de multiplicación $0b = 0$ por ende no puede ser igual a $a$ porque $a \neq 0$. Por lo tanto $a \div 0 = b$ no existe.

\paragraph{Caso II} Ahora suponemos que $a = 0$ e intentamos encontrar un único numero $b$ tal que 

$$0b = a = 0$$

como $0b = 0$ es verdadero para cualquier $b$ por la definición de multiplicación, no hay un \textbf{único} valor de $b$ que satisfaga esta relación. y por lo tanto $0 \div 0$ no está definido. $\blacksquare$

\subsection{Los números primos} Anteriormente hemos evidenciado que los números se pueden construir de manera aditiva, pero también pueden construirse mediante la operación de multiplicación. Esto es mediante los números primos, estos números tienen características muy particulares, por lo que antes de continuar revisaremos algunas definiciones.

\paragraph{Definición} Sean $a,b$ números enteros con $a$ diferente de cero. Decimos que $a$ divide a $b$ si existe un entero $c$ tal que $b=ac$. En tal caso escribimos $a|b$. Decimos también que $a$ es un divisor de $b$ o que $b$ es un múltiplo de $a$. Para indicar lo contrario(que $a$ no divide a $b$) escribimos $a \nmid b$.

\paragraph{Definición} Un entero positivo $p > 1$ se denomina un número primo si tiene exactamente dos divisores positivos a saber: 1 y p. Un entero positivo mayor que 1 que no es primo se denomina compuesto.

\paragraph{Teorema fundamental de la aritmética} Cualquier entero mayor a 1 puede rescribirse como el producto único de numeros primos.

\paragraph{}
Justamente este teorema, reza lo que nos indica la noción de descomponer números, que los números pueden de cierto modo descomponerse hasta sus factores mas fundamentales y que estos son los números primos. Extensas propiedades y problemas aún abiertos, como la conjetura de goldbach\footnote{que indica que todo número par mayor que 2 puede expresarse como la suma de dos números primos $36 = 31 + 5$} siguen abiertos desde antes de la época de Euclides y grandes matemáticos, de la talla de Euler y Gauss han intentado descubrir cuál es la forma de encontrar la distribución de estos números, su respuesta tiene implicaciones mucho mas allá de las matemáticas.

\section{Los números reales}

Aún con todos estos conjuntos se obtienen resultados de operaciones que no es posible expresar con ninguno de los conjuntos anteriores, a estos números se les conoce como irracionales, y en definición, son los números que no se pueden expresar como un cociente entre números enteros(que no son racionales), aquí una breve história de cómo se nombraron los conjuntos obtenido de \cite{patino_duque_1977}:

\paragraph{}
\say{Euclides al hacer una clasificación sistemática de los números los dividió en dos grandes grupos que llamó symmetros(con medida) y asymmetros(sin medida) a cada uno de ellos. Al no encontrar una expresión simbólica adecuada para estos últimos los designó con la voz alogos. Gerardo de Cremona\footnote{Gerardo de Cremona gracias a su curiosidad por los tratados de oriente, estuvo en contacto con múltiples obras árabes que tradujo laboriosamente al latín, como el almagesto de Ptolomeo o la Geometría de Euclides.}, (1114-1187) al traducir un comentario árabe sobre Euclides, tomó los vocablos logos y alogos en su acepción de razón y no de la palabra coomo la quizo Euclides, y empleó erróneamente los términos rationalis e irrationalis para designar unos y otros. Este error se difundió durante toda la edad media hasta llegar a nosotros}.

\paragraph{} retomando nuestra composición del conjunto... A este conjunto le denota con la letra $\mathbb{I}$. Estos números aparecen por primera vez en el teorema de pitágoras.

\paragraph{Teorema de pitágoras} Sean a y b los catetos de un triangulo rectángulo, su hipotenusa c, satisface $a^2 + b^2 = c^2$.

\paragraph{} Veamos que en algunos casos la solución no existe en los racionales, supongamos que $a = b = 1$, por el teorema de pitágoras vemos que

$$c^2 = 1^2 + 1^2 = 2$$

pero no existe ningún número racional mediante el cual al multiplicarse por sí mismo sea igual a 2. Y vamos a demostrarlo.

\paragraph{Demostración} \textit{Adaptada de \cite{brink_1933}} Asumamos que existe un número racional tal que $c = \frac{p}{q}$, donde $p$ y $q$ son enteros, y asuma también que la fracción $\frac{p}{q}$ ha sido simplificada a sus terminos mínimos.

Veremos que esto nos lleva a una contradicción y por lo tanto debe ser falso. Como $c^2 = 2$ tenemos:

$$\frac{p^2}{q^2} = 2 \text{ ó que } p^2 = 2q^2$$

Por ende $p^2$ es un número par, porque contiene el factor $2$. Pero el cuadrado de cualquier número impar es impar. en consecuencia $p$ no puede ser un número impar y debe ser par, por lo que podemos decir que $p = 2n$, donde $n$ es un entero. De la ecuación $p^2 = 2q^2$ tenemos que $(2n)^2 = 2q^2$ o $4n^2 = 2q^2$. Por ende, $2n^2 = q^2$, lo cual muestra que $q^2$ es par y consecuentemente $q$ también es par. Dado que ambos números son pares, el númerador y denominador de $\frac{a}{b}$ contienen el factor 2. Esto contradice nuestra tésis que $\frac{a}{b}$ estaba reducido a su mínima fracción, y demuestra que no existe ningún número racional que elevado al cuadrado sea igual a $2$. $\blacksquare$.

\paragraph{Definición} Los números irracionales, denotados con el símbolo $\mathbb{I}$ son los números que no pueden representarse como un racional, es decir:

$$\mathbb{I} := \{x \in \mathbb{R} | x \neq \mathbb{Q} \}$$

\paragraph{Definición} Los números reales, denotados con el símbolo $\mathbb{R}$ son la unión del conjunto de los números irracionales $\mathbb{I}$ y el conjunto de los números racionales $\mathbb{Q}$, es decir:

$$\mathbb{R} := \mathbb{Q} \cup \mathbb{I} $$

\subsection{Los números trascendentales}

Dentro de los números reales hay dos tipos, los números algebráicos y los números trascendentales.

\paragraph{Definición} Un número algebráico es un número que puede expresarse como la raíz de un polinomio.

\paragraph{Definición} Un número trascendental\footnote{Se desconocen muchos de éstos números y así mismo la prueba de trascendencia se lleva a cabo caso a caso. Por ejemplo una de las prueba de que $\pi$ es un número trascendental fue llevada a cabo por Lindeman en 1882, utilizó técnicas muy similares como las que se utilizaron por Hermite para demostrar la trascendencia de $e$.} es un número que no puede expresarse como la raíz de un polinomio.


\paragraph{}
Cantor tiene una de las demostraciones de las que se deriva que pueden existir "mas" números trascendentales que números algebráicos\cite{niven_1961}.

\section{¿Y estos son todos los números que hay?}

Hasta aquí hemos visto es que los conjuntos se han construído como una necesidad para llenar un vacío que necesariamente requiere de un nuevo tipo de número, lo cierto es que hasta el momento el siguiente conjunto cubre todos los anteriores y nace en necesidad de hallar la raíz de un número negativo. A éste número se le conoce como los complejos $\mathbb{C}$ y dentro de este se encuentran todos los conjuntos anteriormente descritos.

\paragraph{} El símbolo $\sqrt{-1} = i$, es la unidad en los números imaginarios; y la propiedad $i^2 = -1$ es la definición de unidad imaginaria\cite{patino_duque_1977}.

Y más alla, existen otros conjuntos, como los hipercomplejos, los transfinitos y los hiperreales.

\section{Conclusiones}
Las matemáticas como otra rama del pensamiento humano de las que mayor trabajo y esfuerzo ha tenido a lo largo de la historia, son una construcción, pero en particular es una de las construcciones más rigurosas, exige la prueba como una herramienta necesaria para la construción, y lo evidenciamos en algunas partes de éste texto. Los números los utilizamos desde las nociones más intuitivas, como por ejemplo las operaciones básicas en los enteros y en los naturales; y en los racionales encontramos matemáticas que son utilizadas en el día a día al dividir (por ejemplo), pero no siempre es así ¿Cuándo requerimos conocer si un numero es trascendental o no? verlo únicamente de una manera práctica, limita enormemente nuestra capacidad para el descubrimiento y nos impide ver la belleza oculta que tienen, así que desde el imaginario de lo imposible de un estudiante de matemáticas, hace reflexionar está a la espera de ser descubierto.

\nocite{*}

\bibliography{bibliography}

\end{document}

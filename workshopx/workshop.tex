\documentclass{article}

\usepackage{amsmath}
\usepackage{amsfonts}
\usepackage{amssymb}
\usepackage{amsthm}
\usepackage{amsrefs}
\usepackage[spanish, mexico]{babel}
\usepackage[margin=0.5in]{geometry}
\usepackage[utf8]{inputenc}

\begin{document}
	\title{Taller X}
	\author{Leidy C. Sánchez, Miguel A. Gómez}
	\maketitle

\paragraph{1.} Reproduce and explain in full detail all the solutions presented for the problem of a card game, including theone given a congruence.

\paragraph{2.} Reproduce and explain in full datail all the solutions presented for the Han Xing soildiers problem.

\paragraph{3.}  Reproduzca y explique en completo detalle las soluciones presentadas para el problema de los priatas.

\paragraph{Problema de los piratas} Una banda de 17 piratas se apodera de un botín compuesto por monedas de oro de igual valor. Deciden repartirse el botín en partes iguales y dar el resto al cocinero chino. Así, el cocinero recibirá tres monedas. Pero los piratas se pelean entre ellos y seis de ellos mueren en la riña. El cocinero recibiría entonces 4 monedas. Posteriormente ocurre un naufragio y solo 6 piratas, el cocinero y el tesoro se salvan. La nueva repartición dejaría 5 monedas de oro al cocinero ¿Cuál es la fortuna mínima que esperaría el cocinero si decide liquidar al resto de los piratas?

\paragraph{Explicación en detalle de la solución} se construye el sistema de 3 congruencias,

\begin{enumerate}
    \item $x \equiv 3 \pmod{17}$. Esta se construye en base a que originalmente el tesoro se debe repartir en partes iguales entre 17 piratas, y por lo cual quedan 3 monedas de oro para el cocinero.
    \item $x \equiv 4 \pmod{11}$. Esta congruencia se construye en base a los 6 piratas que mueren en la riña, $17-6 = 11$ piratas y que deja como botín al cocinero 4 monedas.
    \item $x \equiv 5 \pmod{6}$. La última congruencia la construímos con los últimos 6 piratas que quedan despues del naufragio y que deja como residuo 5 monedas del botín para el cocinero.
\end{enumerate}

Ahora procedemos a solucionar este sistema de congruencias. rescribimos la primera congruencia como una ecuación diofántica, así:

$$x \equiv 3 \pmod{17} \implies x = 3 + 17s \text{, con } s \in \mathbb{Z}$$

con sustituímos este valor para $x$ en la segunda congruencia y simplificamos:

$$3 + 17s \equiv 4 \pmod{11} \implies 17s \equiv 1 \pmod{11}$$

nos interesa conocer los valores que puede tener $s$, por lo que ahora buscamos un valor que sea múltiplo de 17 y congruente a 1 módulo 11, nótese que:
$$34 = 17(2) - 1 \implies 34 \equiv 1 \pmod{11}$$
ahora multiplicamos por 2: $17(2)s \equiv 1(2) \pmod{11}$, por propiedades de las congruencias podemos afirmar que lo anterior implica que 

$$s \equiv 2 \pmod{11} \implies s = 2 + 11k \text{, con k } \in \mathbb{Z}$$

Realizamos la sustitución en $x = 3 + 17s$,

$$x = 3 + 17(2 + 11k) = 3 + 34 + 187k = 37 + 187k$$

este nuevo valor para $x$ lo sustituímos en la tercera congruencia, luego

$$37 + 187k \equiv 5 \pmod{6}$$


\paragraph{4.} Reproduce and explain in full detail the solutions presented for the eggs problem. And based on it, invent a problem which will be solved with a congruence.

\paragraph{5.} Un entero compuesto $n$ se llama pseudoprimo siempre y cuando $n|2^n - 2$. Se puede observar existen infinitos pseudoprimos, los primeros cuatro 341, 561, 645 y 1105. Demuestre que, si $n$ es un pseudoprimo impar, entonces $M_n = 2^n - 1$ es uno aún mayor.

\begin{proof}
Si $n$ es compuesto, por definición de número compuesto, significa que es múltiplo de algun número primo $p$, es decir, es de la forma $n = dp$ con $d$ y $p \in \mathbb{Z}$.

\paragraph{} Ahora es necesario realizar la demostración del siguiente teorema\footnote{Punto 7, assignment III}.

\paragraph{Teorema} Si $n$ y $d$ son enteros tal que $n|d$, entonces $2^d - 1 | 2^n - 1$.

\begin{proof}
Por definición de divisibilidad, $n = md$, con $m \in \mathbb{Z}$, al reemplazaren la expresión $2^{d} + 1$, tenemos:
	
	$$2^n -1 = 2^{md} - 1$$

al aplicar la identidad $x^k - 1 = (x-1)(x^{k-1} + x^{k-2} + \dots + x + 1)$ en la expresión anterior tenemos que

\begin{align*}
	2^{md} - 1 &= (2^d - 1) ((2^{d})^{m - 1} + (2^{d})^{m - 2} + \dots + 2^d + 1)\\
	&= (2^d - 1) k \text{, con } k = (2^d)^{m-1} + (2^d)^{m-2} + \dots + 2^d + 1
\end{align*}

Y por lo tanto 

$$2^n - 1 = (2^d - 1) - k$$

y  por divisibilidad, esto es equivalente a que $2^d - 1 | 2^n - 1$
\end{proof}

Continuando con la demostración inicial, como $M_n = 2^n -1$, entonces por el teorema demostrado anteriormente, $2^d - 1 | M_n$, como $M_n$ es compuesto si $n$ es compuesto.

\paragraph{} $n|2^n - 2$ se puede rescribir como $2^n \equiv 2 \pmod{n}$ y $2^n - 2 = qn$, con $q \in \mathbb{Z}$. Sin embargo nótese tambien que $2^n - 2 = 2^n - 1 - 1$, que es la forma original de un número de Merssene, y por ende $(2^n - 1) - 1 = M_n - 1 = qn$.

\paragraph{} Por propiedades de los exponentes $2^{M_n - 1} = 2^{qn}$ y $2^{M_n - 1} - 1 = 2^{qn} - 1$. Utilizando la identidad derivadadel teorema binomial\footnote{$x^k - 1 = (x-1)(x^{k-1} + x^{k-2} + \dots + x + 1)$} en $2^{qn} - 1$:

\begin{align*}
2^{qn} - 1 &= (2^n - 1)((2^n)^{q-1} + (2^n)^{q-2} + \dots + 2^n + 1)\\
&= M_n((2^n)^{q-1} + (2^n)^{q-2} + \dots + 2^n + 1)
\end{align*}

y como $2^{mn -1} - 1 = 2^{qn} - 1$, podemos divir la parte derecha de la igualdad entre $M_n$ y por ende $2^{M_n - 1} - 1 \equiv 0 \pmod{M_n} \implies 2^{M_n - 1} \equiv 1 \pmod{M_n}$ y por propiedades de las congruencias $2(2^{M_n -1}) \equiv 2(1) \pmod{M_n} \implies 2^{M_n} \equiv 2 \pmod{M_n}$ y que podemos rescribir como

$$2^{M_n} - 2 \equiv 0 \pmod{M_n} \implies M_n | 2^{M_n} - 2$$

por lo que $M_n$ es pseudoprimo $M_n = 2^n - 1$, $M_n$ es mayor que $n$.
\end{proof}
	
\paragraph{6.} Las tres apareiciones mas recientes del cometa Halley fueron en los años 1985, 1910 y 1986; la próxima será en 2061. demuestre que

$$1835^{1910} + 1986^{2061} \equiv 0 \pmod{7}$$

\begin{proof}
$1835 = (7)(262) + 1$, por lo tanto $1835 \equiv 1 \pmod{7}$ y por propiedades de las congruencias $1835^{1910} \equiv 1^{1910} \pmod{7}$ es equivalente a $1835^{1910} \equiv 1 \pmod{7}$.

\paragraph{} Por otra parte, el número 1986 puede escribirse como $(7)(283) + 5$, por lo tanto $1986 \equiv 5 \pmod{7}$ y por propiedades de las congruencias $1986^{2061} \equiv 5^{2061} \pmod{7}$ que es equivalente a $1986^{2061} \equiv (5^3)^{687} \pmod{7}$.

\paragraph{}Nótese que $5^2 = 125 = 126 - 1$, es decir que $5^3 \equiv -1 \pmod{7}$, por propiedades de congruencias $(5^3)^{687} \equiv (-1)^{687} \pmod{7}$, que es equivalente a $(5^3)^{687} \equiv -1 \pmod{7}$ y por transitividad equivalente a que $1986^{2061} \equiv -1 \pmod{7}$. Con lo anterior por propiedades de las congruencias se sigue que

\begin{align*}
1835^{1910} + 1986^{2061} &\equiv 1 + (-1) \pmod{7}\\
&\equiv 0 \pmod{7}
\end{align*}

y que era lo que queríamos demostrar.

\end{proof}

\nocite{*}

\bibliography{bibliography}

\end{document}
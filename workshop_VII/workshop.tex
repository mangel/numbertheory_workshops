\documentclass{article}

\usepackage{amsfonts}
\usepackage{amsmath}
\usepackage{amssymb}
\usepackage{amsthm}
\usepackage[utf8]{inputenc}

\title{Workshop VIII}
\author{Cesar A. Lara y Miguel A. Gomez}


\begin{document}
	\maketitle

\paragraph{}1. Use the theory of congruences to verify that

$$89|2^{44} - 1 \text{ and } 97|2^{48} - 1$$

\paragraph{Proof} $89|2^{44} - 1$. We first establish a relationship between $89$ and a number that is a power of $2$:

$$2^8 = 256 = (3)(89)-11$$

By definition of congruences, this also can be re-written into $2^{8} \equiv (-11)\pmod{ 89}$, and because of the symmetry property of congruences, this also implies that:

$$-11 \equiv {2^{8}}\pmod{89}$$

By the multiplication property\footnote{Given some element $c$ if $a \equiv b\pmod{n}$, then $ac \equiv bc \pmod{n}$.} we obtain:

\begin{align*}
	8(-11) &\equiv (2^8)8 \pmod{89}\\
	(-88) &\equiv (2^{8}) (2^{3}) \pmod{89}\\
	(-88) &\equiv (2^{11})\pmod{89}
\end{align*}

We observe the relationship between $88$ and $89$ in terms of integer division:

$$89 = 88 +1 \implies 88 \equiv 1 \pmod{89}$$.

Multipliying both sides by $-1$ we have 
$$-88 \equiv -1 \pmod{89}$$

and by the transitivity property, we can establish a link between two unseemly related congruences:

\begin{align*}
	-88 \equiv -1 \pmod{89} \land (-88) \equiv (2^{11})\pmod{89} \implies 2^{11} \equiv -1 \pmod{89}
\end{align*}

Then, by the power property\footnote{For a given positive integer k, $a^k \equiv b^k \pmod{n}$}:

\begin{align*}
	(2^{11})^{4} &\equiv (-1)^4 \pmod{89}\\
	2^{44} &\equiv 1 \pmod{89}\\
\end{align*}

And lastly we substract 1 from both sides of the congruence to finally obtain

$$2^{44} - 1 \equiv 0 \pmod{89}$$

By which we prove, that $89|2^{44} - 1$. \qed

\paragraph{Proof} $97|2^{48} - 1$. We firs state one of Fermat's Theorem.

\subparagraph{Theorem} If $p$ is a prime number and $(a,b) = 1$, then:

$$a^{p-1} \equiv{ 1 \pmod{p} }$$

\paragraph{} Continuing our initial proof we obtain:

$$97 \not{|} 2 \implies 2^{96} \equiv{1\pmod{97}}$$

\begin{align*}
    2^{96} - 1 &= (k)(97)\\
    (2^{48} + 1)(2^{48} - 1) &= (k)(97)
\end{align*}

We must verify now that this expression must be divisible by $97$, which is true, because

$$97 = 2(48) + 1 \implies 2 \equiv 1 \pmod{97}$$

Then by the power property

$$2^{48} \equiv 1^{48} \pmod{97} \implies 2^{48} \equiv 1 \pmod{97}$$

By the adding property we obtain

$$2^{48} - 1 \equiv 1 -1 \pmod{97} \implies 2^{48} - 1 \equiv 0 \pmod{97}$$

Which implies that $97$ indeed divides $2^{48} - 1$. \qed

\newpage

\paragraph{}2. If p is a prime satisfiying $n<p<2n$, show that

$$\binom{2n}{n} \equiv 0\pmod{p} $$

\paragraph{Proof}
\begin{align*}
  \binom{2n}{n} = \dfrac{(2n)!}{(2n-n)!n!} &= \dfrac{(2n)(2n-1)(2n-1)(\dots) (p)(\dots)(n+1)(n!)}{n!n!}\\
  &= \dfrac{(2n)(2n-1)(2n-2)(2n-3)(2n-4)(\dots)(p)(\dots)(n+1)}{n(n-1)(n-2)(\dots)(1)}
\end{align*}

Patterns of common factors can be finded in the expression:

$$2n-6=2(n-3), 2n-8=2(n-4), 2n-5 = 2(n-5)$$

By which we can simplify into:

$$(2n-1)(2n-3)(2n-5).....p.....(n+1)$$
Then:

$$ \binom{2n}{n} \equiv{0\pmod{p}}$$

\qed

\paragraph{}3. Find the remainder when $4444^{444}$ is divided by 9. Hint: Observe that $2^{3} \equiv -1\pmod{9}$.

\paragraph{Proof} We start by exploring the number $4444^{444}$ in terms of is prime factors.

$$4444 = (2)(2)(11)(101)$$
$$444 = (2)(2)(3)(37)$$

If we inspect these numbers mod 9, we get respectively:

\begin{align*}
  2^2 &\equiv 2^2 \pmod{9} \implies 2 \equiv 2 \pmod{9}\\
  101 &\equiv 2 \pmod{9}\\
  11 &\equiv 2 \pmod{9}
\end{align*}

We start the construction of symmetries between the prime decomposition of $44444$, we begin with:

$$101 \equiv 2 \pmod{9}$$

By the multiplication property, we multiply by $11$ (which is another prime factor of our number), then:

$$(101)(11) \equiv 22 \pmod{9}$$

Be congruent by $22$ is the same as being congruent by $4$:
$$22 = 9(2) + 4 \implies 22 \equiv 4 \pmod{9}$$

Then,

$$(101)(11) \equiv 4 \pmod{9}$$

We multiply again by $2^2$ the congruence:

$$(4)(101)(11) \equiv 4(4) \pmod{9} \implies (4)(101)(11) \equiv 16 \pmod{9}$$

We evaluate $16$ divisibility by $9$:

$$16 = 9(1) + 7 \implies 16 \equiv 7 \pmod{9}$$

Then,

$$4444 = (4)(101)(11) \equiv 7 \pmod{9}$$

And by the power property

$$4444^{444} = ((4)(101)(11))^{444} \equiv 7^{444} \pmod{9}$$

Then $7^{444}$ is the remainder, repeating the same procedure we see that

$$7 = 9(0) + 7 \implies 7 \equiv 7 \pmod{9}$$

But also

$$7 = 9(1) - 2 \implies 7 \equiv -2 \pmod{9}$$

Then,

$$7^{444} \equiv (-2)^{444} \pmod{9}$$

Which is the same as

$$7^{444} \equiv (2)^{444} \pmod{9}$$

and the same as

$$7^{444} \equiv (2^3)^{(4)(37)} \pmod{9}$$

By the hint we know that

$$2^{3} \equiv -1 \pmod{9}$$

then

$$2^{444} = (2^{3})^{(4)(37)} \equiv (-1)^{(4)(37)} \pmod{9}$$

Which is the same as

\begin{align*}
  2^{444} = (2^{3})^{(4)(37)} &\equiv (1)^{(4)(37)} \pmod{9}\\
  &\equiv 1 \pmod{9}
\end{align*}

Then,

$$7^{444} \equiv 1 \pmod{9}$$

and at last,

$$4444^{444} \equiv 1 \pmod{9}$$

The remainder is $1$. \qed

\end{document}
\documentclass{article}

\usepackage{amsfonts}
\usepackage{amsmath}
\usepackage{amssymb}
\usepackage{amsthm}
\usepackage[utf8]{inputenc}

\title{Workshop VIII}
\author{Cesar A. Lara y Miguel A. Gomez}


\begin{document}
	\maketitle

\paragraph{}1. Use the theory of congruences to verify that

$$89|2^{44} - 1 \text{ and } 97|2^{48} - 1$$

\paragraph{Proof} $89|2^{44} - 1$. We first establish a relationship between $89$ and a number that is a power of $2$:

$$2^8 = 256 = (3)(89)-11$$

By definition of congruences, this also can be re-written into $2^{8} \equiv (-11)\pmod{ 89}$, and because of the symmetry property of congruences, this also implies that:

$$-11 \equiv {2^{8}}\pmod{89}$$

By the multiplication property\footnote{Given some element $c$ if $a \equiv b\pmod{n}$, then $ac \equiv bc \pmod{n}$.} we obtain:

\begin{align*}
	8(-11) &\equiv (2^8)8 \pmod{89}\\
	(-88) &\equiv (2^{8}) (2^{3}) \pmod{89}\\
	(-88) &\equiv (2^{11})\pmod{89}
\end{align*}

We observe the relationship between $88$ and $89$ in terms of integer division:

$$89 = 88 +1 \implies 88 \equiv 1 \pmod{89}$$.

Multipliying both sides by $-1$ we have 
$$-88 \equiv -1 \pmod{89}$$

and by the transitivity property, we can establish a link between two unseemly related congruences:

\begin{align*}
	-88 \equiv -1 \pmod{89} \land (-88) \equiv (2^{11})\pmod{89} \implies 2^{11} \equiv -1 \pmod{89}
\end{align*}

Then, by the power property\footnote{For a given positive integer k, $a^k \equiv b^k \pmod{n}$}:

\begin{align*}
	(2^{11})^{4} &\equiv (-1)^4 \pmod{89}\\
	2^{44} &\equiv 1 \pmod{89}\\
\end{align*}

And lastly we substract 1 from both sides of the congruence to finally obtain

$$2^{44} - 1 \equiv 0 \pmod{89}$$

By which we prove, that $89|2^{44} - 1$. \qed

\paragraph{Proof} $97|2^{48} - 1$. We firs state one of Fermat's Theorem.

\subparagraph{Theorem} If $p$ is a prime number and $(a,b) = 1$, then:

$$a^{p-1} \equiv{ 1 \pmod{p} }$$

\paragraph{} Continuing our initial proof we obtain:

$$97 \not{|} 2 \implies 2^{96} \equiv{1\pmod{97}}$$

\begin{align*}
    2^{96} - 1 &= (k)(97)\\
    (2^{48} + 1)(2^{48} - 1) &= (k)(97)
\end{align*}

We see now that this expression must be divisible by $97$. \qed

\paragraph{}2. If p is a prime satisfiying $n<p<2n$, show that

$$\binom{2n}{n} \equiv 0\mod{p} $$

\paragraph{}3. Find the remainder when $4444^{444}$ is divided by 9. Hint: Observe that $2^{3} \equiv -1(\mod{9})$
	
\end{document}